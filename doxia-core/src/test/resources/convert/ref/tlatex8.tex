\batchmode
\newcommand{\psectioni}[1]{\section*{#1}}
\newcommand{\psectionii}[1]{\subsection*{#1}}
\newcommand{\psectioniii}[1]{\subsubsection*{#1}}
\newcommand{\psectioniv}[1]{\paragraph*{#1}}
\newcommand{\psectionv}[1]{\subparagraph*{#1}}
\newcommand{\ptitle}[1]{\title{#1}}
\newcommand{\pauthor}[1]{\author{#1}}
\newcommand{\pdate}[1]{\date{#1}}
\newcommand{\pmaketitle}{\maketitle}
\newenvironment{plist}{\begin{itemize}}{\end{itemize}}
\newenvironment{pnumberedlist}{\begin{enumerate}}{\end{enumerate}}
\newcommand{\pdef}[1]{\textbf{#1}\hfill}
\newenvironment{pdefinitionlist}
{\begin{list}{}{\settowidth{\labelwidth}{\textbf{999.}}
                \setlength{\leftmargin}{\labelwidth}
                \addtolength{\leftmargin}{\labelsep}
                \renewcommand{\makelabel}{\pdef}}}
{\end{list}}
\newenvironment{pfigure}{\begin{center}}{\end{center}}
\newcommand{\pfigurecaption}[1]{\\* \vspace{\pparskipamount}
                                \textit{#1}}
\newenvironment{ptable}{\begin{center}}{\end{center}}
\newenvironment{ptablerows}[1]{\begin{tabular}{#1}}{\end{tabular}}
\newenvironment{pcell}[1]{\begin{tabular}[t]{#1}}{\end{tabular}}
\newcommand{\ptablecaption}[1]{\\* \vspace{\pparskipamount}
                               \textit{#1}}
\newenvironment{pverbatim}{\begin{small}}{\end{small}}
\newsavebox{\pbox}
\newenvironment{pverbatimbox}
{\begin{lrbox}{\pbox}\begin{minipage}{\linewidth}\begin{small}}
{\end{small}\end{minipage}\end{lrbox}\fbox{\usebox{\pbox}}}
\newcommand{\phorizontalrule}{\begin{center}
                              \rule[0.5ex]{\linewidth}{1pt}
                              \end{center}}
\newcommand{\panchor}[1]{\index{#1}}
\newcommand{\plink}[1]{#1}
\newcommand{\pitalic}[1]{\textit{#1}}
\newcommand{\pbold}[1]{\textbf{#1}}
\newcommand{\pmonospaced}[1]{\texttt{\small #1}}
\newcommand{\pfiguregraphics}[1]{\includegraphics{#1.eps}}

\documentclass[letterpaper]{article}
\usepackage[english]{babel}
\usepackage{graphicx}
\usepackage{ifthen}
\usepackage[T1]{fontenc}
\usepackage[latin1]{inputenc}

\pagestyle{plain}

\newlength{\pparskipamount}
\setlength{\pparskipamount}{1ex}

\begin{document}

\newpage

\psectioni{The docc\panchor{docc} utility}

See also the docx utility.

docc is auto\symbol{45}documented. Simply type \pmonospaced{docc} to list its
generic options:

\begin{pverbatim}
\begin{verbatim}
$ docc
usage: docc target_format [generic_options] [format_options] {input_file}+

target_format: latex, man, html, rtf or href (pseudo-format)

Note that PTF comments are automatically extracted from C/C++ source files
using docx if the extension of the input file name is .[hH]* or .[cC]*. Tcl
source files are supported too if the extension of the input file name is .tcl
or .TCL.

generic_options:
-c 
Each input file is individually translated to a stand-alone
documentation file (like 'cc -c').
Default: all input files are translated to a single output file.
-o <%255s>
Specify the name of the output file (like 'cc -o').
Default: the basename of the input file+a format specific suffix.
-href (-h) <%255s>
Specify the name of the hypertext references file to be loaded
(generated during a first pass using the 'href' pseudo-format).
Default: none.
-sed (-s) <%255s>
Specify the name of a file which contains sed commands. These
sed commands are applied to all PTF source files (even if
automatically extracted using docx) just before their translation
to the target format.
Default: none.
-borders (-b) 
Tell docx to add borders around extracted code.
Default: no borders.

Type 'docc target_format' to list the options related to
target_format.
\end{verbatim}
\end{pverbatim}

Then, for example, type \pmonospaced{docc rtf} to list the options related to
the RTF format:

\begin{pverbatim}
\begin{verbatim}
$ docc rtf
-linear (-li) 
The output RTF file does not contain hypertext links a la WinHelp.
Default: non linear (WinHelp).
-adobefonts (-af) 
Use Adobe fonts (Helvetica, Times, etc).
Default: Windows fonts (Arial, Time New Roman, etc).
\end{verbatim}
\end{pverbatim}

\psectionii{Examples}

\psectioniii{LaTeX}

PTF comments are directly extracted from C/C++ sources files.

\begin{pverbatim}
\begin{verbatim}
$ cd tmp
$ docc latex -dc article -dco a4paper \
-ti 'The PTF Format and Related Utilties' \
-o ptf.tex ptf.txt docc.cc docx.cc
$ ls
ptf.tex
\end{verbatim}
\end{pverbatim}

\psectioniii{troff \symbol{45}man}

docc is run on docc.cc and then on docx.cc to generate two man pages with
different headers.

\begin{pverbatim}
\begin{verbatim}
$ cd tmp
$ docc man -c -ti docc -se 1 -he 'Text Utilities' -fo Pixware docc.cc
$ docc man -c -ti docx -se 1 -he 'Text Utilities' -fo Pixware docx.cc
$ ls
docc.man        docx.man
\end{verbatim}
\end{pverbatim}

\psectioniii{HTML}

The hardest format to generate is HTML because in this case you must run docc
twice, example:

\begin{pdefinitionlist}

\item[\mbox{First pass}] Resolve all hypertext links and save them in a file
called \pmonospaced{hrefs}. To do this, you need to use the \pitalic{href}
pseudo\symbol{45}format.

\begin{pverbatim}
\begin{verbatim}
$ cd tmp
$ docc href -o hrefs ptf.txt docc.cc docx.cc
$ ls
hrefs
\end{verbatim}
\end{pverbatim}

\item[\mbox{Second pass}] For each input file, generate an HTML file (the
\symbol{45}c option). Load file \pmonospaced{hrefs} (the \symbol{45}h hrefs
option) to be able to resolve hypertext links.

\begin{pverbatim}
\begin{verbatim}
$ docc html -br -h hrefs -c ptf.txt docc.cc docx.cc
$ ls
docc.html        hrefs           previous.gif
docx.html        next.gif        ptf.html
\end{verbatim}
\end{pverbatim}

The \symbol{45}br option is used to chain the generated HTML files not only
through hypertext links but also sequentially (i.e. like the pages of a book).

\end{pdefinitionlist}

\psectioniii{RTF}

Always use \symbol{45}o when generating non\symbol{45}linear RTF format
(WinHelp) to get a single .rtf file and a single .hpj (help project) file.

\begin{pverbatim}
\begin{verbatim}
$ cd tmp
$ docc rtf -o docc.rtf ptf.txt docc.cc docx.cc
$ ls
docc.hpj        docc.rtf
\end{verbatim}
\end{pverbatim}

Unlike the HTML format, the generation of WinHelp files does not require
running docc twice. The hypertext links are resolved by the help compiler
(example \pmonospaced{hc31.exe}) that you must run under DOS on the generated
.hpj file.

\begin{pverbatim}
\begin{verbatim}
C> hc31 docc
\end{verbatim}
\end{pverbatim}

\end{document}

